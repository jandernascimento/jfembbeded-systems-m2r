\documentclass[a4paper]{article}

\usepackage{listings}
\usepackage{color}
\usepackage{graphicx}
\usepackage{tikz}
\usepackage{multicol}

\usepackage[top=2.0cm, bottom=2.0cm, left=2.0cm, right=2.0cm]{geometry}

\begin{document}

\title{Embbeded Systems - Exercise 04}

\author{Antonios \and Jander Nascimento}

\maketitle

\section*{Model checking and fairness}

\subsection*{Question 01}

$
S_f=\nu Z.EX \land_{c E C}(E Z \cup (Z \land C)) \newline
\rightarrow \nu Z.EX((E Z \cup ( Z \land c_1)) \land (E Z \cup (Z \land c_2))) \newline
\newline
Z_0=S \newline
Z_1=EX((E Z_0 \cup (Z_0 \land c_1) \land (E Z_0  \cup (Z_0 \land c_2)) \newline
\rightarrow EX((EF \{7\}) \land (EF\{4,9\}))
\rightarrow EX(\{1-7\} \land \{1-5,8-9\})
\rightarrow EX(\{1-5\})=\{2-5\}
\newline
Z_2=EX((E Z_1 \cup (Z_1 \land c_1) \land (E Z_1  \cup (Z_0 \land c_2)) \newline
\rightarrow EX((E Z_1 \cup (Z_1 \land c_1) \land (E Z_1 \cup (Z_1 \land C_2))) \newline
\rightarrow EX((E\{2-5\} \cup \emptyset) \land (E \{2-5\} \cup \{4\})) \newline
\rightarrow EX(\emptyset \land (E \{2-5\} \cup \{4\})) = \emptyset \newline
Z_3=EX((E \emptyset \cup \emptyset) \land (E \emptyset \cup \emptyset))=\emptyset
$

As the development do not converge we assume that there are no fair paths in the \emph{Kripke} structure.

\subsection*{Question 02}

$
S_f=\nu Z. EX (EZ \cup (Z \land \{b,c\} \newline
Z_0=S \newline
Z_1=EX (EZ.\cup(Z_0 \land \{b,c\})) \newline
\rightarrow EX (E\,S\,\cup \{b,c\}) \newline
\rightarrow EX (E\,S\,\cup \{b,c\} = EF \{b,c\}=S)  \newline
\newline
Z\,EX(S)=S \newline
S_f=S \newline
$

$
K,a \models AF\,AG\,p \newline
A_{c_1}G_p=\neg E_{c_1}F \neg p = \neg E{c_1}F \{ b \} \newline
\newline
\rightarrow E_{b,c}F \{b\}=\nu Z. ( \{b\} \land S_f) \lor EX\,Z
\rightarrow \nu Z. (\{b\}) \lor EX\,Z \newline
\newline
Z_0=\emptyset \newline
Z_1=\{b\} \lor EX \emptyset = \{b\} \newline
Z_2=\{b\} \lor EX \{b\} = \{a,b\} \newline
Z_3=\{b\} \lor EX \{a,b\}=\{a,b\}=Z_2 \newline
$

So, $A_{c_1}Gp={a,b}$

$
K,a \models A_{c_1}F\{a,b\}=\neg E_{c_1}G \neg \{a,b\} = \neg E_{c_1}G \{c\}
\rightarrow E_{b,c}G\{c\}=\nu Z.\{c\} \land EX (E \, Z \, \cup (Z \land \{b,c\}))
\newline
Z_0=S
Z_1=\{c\} \land EX (E \, Z_0 \cup Z_0 \land \{b,c\}))
\rightarrow \{c\} \land EX (E \, S \cup \{b,c\})=\{c\} \land \{EX\,EF\{b,c\}
\rightarrow \{c\} \land EX S = \{c\}
Z_2=\{c\} \land EX(E\,Z_1 \cup (Z_1 \land \{b,c\}))
\rightarrow \{c\} \land EX(E\,\{c\} \cup \{c\})=\{c\}=Z_1 
$

$
\neg E_{b,c}G\{c\}=\{a,b\} \Rightarrow [[A_c\,F\,AG\,p]]_k=\{a,b\}
K,a \models AF\,AG\,p
c_1=\{b,c\}
$

\subsection*{Question 03}

Propositions without fairness:

For \emph{mutual exclusion}:
\begin{lstlisting}
AG(!(ph0.self=EATING * ph1.self=EATING));
AG(!(ph1.self=EATING * ph2.self=EATING));
AG(!(ph2.self=EATING * ph3.self=EATING));
AG(!(ph3.self=EATING * ph0.self=EATING));
\end{lstlisting}

For \emph{progressdeadlock}:
\begin{lstlisting}
!EF(AG(ph0.self=HUNGRY));
!EF(AG(ph1.self=HUNGRY));
!EF(AG(ph2.self=HUNGRY));
!EF(AG(ph3.self=HUNGRY));
\end{lstlisting}

For \emph{starvation}:
\begin{lstlisting}
!EF(EG(ph0.self=HUNGRY));
!EF(EG(ph1.self=HUNGRY));
!EF(EG(ph2.self=HUNGRY));
!EF(EG(ph3.self=HUNGRY));
\end{lstlisting}

Propositions for fairness:
\begin{lstlisting}
!(ph0.self=EATING);
!(ph1.self=EATING);
!(ph2.self=EATING);
!(ph3.self=EATING);
\end{lstlisting}

\section*{Bisimulation, Simulation}

\subsection*{Question 04}

\subsection*{Question 05}

\subsection*{Question 06}

\end{document}
